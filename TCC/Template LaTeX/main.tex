\documentclass[12pt]{article}

\usepackage{sbc-template}

\usepackage{graphicx,url}
\usepackage{xcolor}
\usepackage{soul}
\usepackage[normalem]{ulem}

%\usepackage[brazil]{babel}   
\usepackage[utf8]{inputenc}  
\usepackage{booktabs, multirow}

\sloppy

\title{Título do Projeto de Pesquisa que será desenvolvido durante o TCC}

\author{Autor1 Sobrenome\inst{1}, Autor2 Sobrenome\inst{1}, Autor3 Sobrenome\inst{1}, \\Autor4 Sobrenome\inst{1}, Orientador\inst{1,2}}

\address{Sistemas de Informação\\
Faculdade de Computação e Informática\\
Universidade Presbiteriana Mackenzie\\
São Paulo -- SP -- Brasil
\nextinstitute
  Ciência da Computação\\
  Faculdade de Computação e Informática\\
  Universidade Presbiteriana Mackenzie\\
  São Paulo -- SP -- Brasil
  \email{\{autor1,autor2,autor3,autor4\}@mackenzista.com.br,
  orientador@mackenzie.br}
}

\begin{document} 

\maketitle
     
\begin{resumo} 
  Este meta-projeto apresenta a estrutura e estilos utilizados para a elaboração do projeto de pesquisa a ser entregue na disciplina de Metodologia da Pesquisa em Computação ministrada na Faculdade de Computação e Informática (FCI) da Universidade Presbiteriana Mackenzie (UPM). O(s) autor(es) deve(m) dispor em até 10 linhas, na versão final do projeto de TCC, o resumo com o tema, objetivo da pesquisa, abordagem teórico-metodológica e resultados esperados.
\end{resumo}
\begin{palavraschave}
  Três a seis unitermos separados por vírgulas.  
\end{palavraschave}

\section{Introdução}

A introdução estabelece uma contextualização fundamentada do tema, assim como apresenta o problema de pesquisa (questão a ser respondida pelo TCC), o objetivo geral (meta a ser alcançada, sendo sua apresentação feita com verbo no infinitivo) e os objetivos específicos (etapas de trabalho para atingir o objetivo geral, também apresentados com verbos no infinitivo dado que traduzem ações). Além disso, trata da justificativa (elementos e referências bibliográficas que mostram a relevância da investigação, assim como contribuições decorrentes do trabalho proposto).

Finalmente, ainda são dispostas a hipótese (afirmação provisória a ser confirmada/refutada durante o TCC) e a organização do documento (descrição das partes constitutivas).

\section{Referencial Teórico}
Disposição de 10-12 referências bibliográficas completas relacionadas ao tema de pesquisa (conforme o padrão da SBC de normatização científica – cuja URL de acesso se encontra na última seção deste documento, denominada de “Referências Bibliográficas”). Cada referência deve descrever, em três a cinco linhas, a justificativa da pertinência da bibliografia no contexto do estudo proposto.

\section{Materiais e Métodos}
Apresentação das etapas e dos materiais (hardwares/softwares etc.) que serão utilizados ao longo da pesquisa, assim como as justificativas pertinentes (embasadas por referências bibliográficas). Inclusive, também é necessário classificar a pesquisa conforme a natureza (básica ou aplicada), abordagem (quantitativa, qualitativa, mista ou quali-quanti), finalidade (exploratória, descritiva, explicativa, metodológica) e meio (bibliográfica, documental, estudo de caso, experimental, pesquisa de campo etc.).

Cabe destacar que as imagens presentes no documento devem possuir, conforme exemplo da Figura 1, a resolução de 600dpi. 

\begin{figure}[h]
\centering
\includegraphics[width=0.4\textwidth]{fci.png}
\caption{Marca da FCI-UPM}
\end{figure}

Já a listagem das referências completas que serão estudadas ao longo da pesquisa deve ser apresentada no final do documento (sendo as informações apresentadas no arquivo ``\texttt{sbc-template.bib}''. Ao longo do texto, as referências fazem uso do comando  \textbackslash\texttt{cite\{chamada-da-referência\}}, resultando, por exemplo, em \cite{boulic:91}, \cite{smith:99} e \cite{knuth:84}.

\section{Cronograma}
Disposição das etapas que serão realizadas [entregue no Projeto Parcial], bem como a duração de cada uma em meses.

\begin{table}[!htp]\centering
\caption{Cronograma para o Desenvolvimento do TCC}\label{tab: }
\scriptsize
\begin{tabular}{lr|r|r|r|r|r|r|r|r|r|r|rr}\toprule
\multirow{2}{*}{\textbf{ATIVIDADE}} &\multicolumn{12}{c}{\textbf{MÊS}} \\\midrule
&1 &2 &3 &4 &5 &6 &7 &8 &9 &10 &11 &12 \\
Levantamento Bibliográfico Complementar & & & & & & & & & & & \\
Atividade 2 & & & & & & & & & & & \\
Atividade 3 & & & & & & & & & & & \\
Atividade 4 & & & & & & & & & & & \\
Atividade 5 & & & & & & & & & & & \\
Atividade 6 & & & & & & & & & & & \\
Atividade 7 & & & & & & & & & & & \\
Atividade 8 & & & & & & & & & & & \\
\bottomrule
\end{tabular}
\end{table}

\bibliographystyle{sbc}
\def\refname{Referências Bibliográficas}
\bibliography{sbc-template}

\end{document}
